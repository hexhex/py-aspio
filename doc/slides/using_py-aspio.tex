% vim:ft=tex:
%
\RequirePackage[l2tabu, orthodox]{nag}
\documentclass[%
beamer,%
english,%
% notes=only,% uncomment to print only notes
10pt,%
]{beamer}

\usepackage[utf8]{inputenc}
\usepackage[T1]{fontenc}
\usepackage{babel}
\usepackage{amsmath, amssymb}
\usepackage{listings}
\usepackage{lstautogobble}
\usepackage{graphicx}
% \usepackage{caption}
% \usepackage{subcaption}
\usepackage{hyperref}
\usepackage{booktabs} % Allows the use of \toprule, \midrule and \bottomrule in tables
\usepackage{pifont}
\usepackage{soul}

% Turn off navigation symbols in the bottom right corner
\setbeamertemplate{navigation symbols}{}

% \usepackage{pgfpages}
% \setbeameroption{show notes on second screen}

\useoutertheme{infolines}
\useinnertheme{rectangles}
\usecolortheme{dolphin}
\usecolortheme{rose}

\setbeamertemplate{footline}[frame number]		% footline displays only frame number
\beamertemplatenavigationsymbolsempty	% deactivates navigation bar

\usefonttheme[onlymath]{serif}
% typeset math in times
\usepackage{times}
% typeset text in helvetica
\usepackage[scaled]{helvet}

\title{Using py-aspio}
% \author{}
% \date{}

% TODO:
% Move ad-hoc settings on listings environments to the global setting
% (problem: how to reduce font size of listings without affecting inline listings?)
% => use \lst@ifdisplaystyle\color{blue}\fi -- see http://tex.stackexchange.com/a/268971
% Also, add keywords for the annotation language
\lstset{showstringspaces=false,
    % basicstyle=\ttfamily\footnotesize,
    % basicstyle=\footnotesize,
    keywordstyle=\bfseries
}

\definecolor{darkgreen}{rgb}{0,0.5,0}

\newcommand{\blue}[1]{{\color{blue}#1}}
\newcommand{\red}[1]{{\color{red}#1}}
\newcommand{\orange}[1]{{\color{orange}#1}}
\newcommand{\darkgreen}[1]{{\color{darkgreen}#1}}


\AtBeginSection[]
{
   \begin{frame}
       \frametitle{Outline}
       \tableofcontents[currentsection]
   \end{frame}
}


\begin{document}

\setlength{\abovedisplayskip}{0pt plus 3pt minus 9pt}
\setlength{\abovedisplayshortskip}{0pt plus 3pt minus 9pt}
\setlength{\belowdisplayskip}{0pt plus 3pt minus 9pt}
\setlength{\belowdisplayshortskip}{0pt plus 3pt minus 9pt}

\begin{frame}
    \titlepage{}
\end{frame}


\section{Motivation and General Approach}

\begin{frame}
    \frametitle{Motivation}

    \begin{block}{Answer Set Programming}
        \begin{itemize}
            \item Answer Set Programming (ASP) is a declarative programming paradigm~\cite{gelf-lifs-91}.
            \item Applications:~\emph{workforce management}~\cite{DBLP:journals/tplp/RiccaGAMLIL12}, \emph{generating holiday plans for tourists}~\cite{DBLP:conf/lpnmr/IelpaILR09},
                for many others cf.~\cite{DBLP:journals/aim/ErdemGL16}.
        \end{itemize}
    \end{block}

    \begin{alertblock}{Limitations}
        \begin{itemize}
            \item Typical end-user applications contain components which cannot be (easily) solved in ASP:
                \begin{itemize}
                    \item graphical user interfaces
                    \item presentation of results
                    \item interfaces to data sources
                    \item etc.
                \end{itemize}
            \item Realizing such components is in the domain of traditional object-oriented (OOP) languages.
        \end{itemize}
    \end{alertblock}
\end{frame}

\begin{frame}
    \frametitle{Motivation}

    \begin{block}{Typical approach}
        \begin{itemize}
            \item Use ASP programs as \blue{components of a larger application}.
            \bigskip
            \item The ASP program solves the \blue{core computational problem}, while other components are implemented in an \blue{object-oriented language}.
            \bigskip
            \item To this end, object-oriented code
                \begin{enumerate}
                    \item adds input as facts,
                    \item evaluates the ASP program, and
                    \item interprets the answer sets.
                \end{enumerate}
            \bigskip
            \item \textcolor{red}{But:} An implementation from scratch is similar for most applications $\Rightarrow$ repetitive work.
        \end{itemize}
    \end{block}
\end{frame}

\begin{frame}
    \frametitle{Evaluating ASP Programs from Object-Oriented Code}

    \begin{block}{Overview}
        \begin{itemize}
            \item We want to use ASP programs similarly to \blue{subprocedures}: \\
                an ASP program $P$ performs a computation over input parameters $v_1, \ldots, v_n$, and each answer set should correspond to a solution object.
                % $\Rightarrow \blue{P.\mathit{solve}(v_1,\dots,v_n)}$ should return a set of objects (=problem solutions).
            \bigskip
            \item Approach: the ASP program is \blue{annotated} with \blue{input/output specifications}. \\
                Annotations are added as \blue{special comments} of form \blue{\tt \%!} to the ASP code.
            \bigskip
            \item \blue{py-aspio} is a Python library for \blue{evaluating} (``calling'')
                such an annotated program from Python code:
                \begin{itemize}
                    \item it takes an \blue{annotated ASP program} and a \blue{list of input parameters} (objects) as input, and
                    \item returns a \blue{set of objects} (corresponding the results of the ASP program).
                \end{itemize}
        \end{itemize}
    \end{block}
\end{frame}

% \begin{frame}
%     \frametitle{Evaluating ASP Programs from Object-Oriented Code}

%     \begin{block}{Evaluation}
%         More precisely, the interpreter library performs the following tasks:
%         \begin{enumerate}
%             \item
%                 Parameters $v_1,\dots,v_n$ are converted to facts according to \blue{input specification $\iota$}.
%             \item
%                 These facts along with the ASP program $P$ are passed to the ASP solver.
%             \item
%                 The answer sets are mapped to objects $O$ according to \blue{output specification $\omega$}.
%         \end{enumerate}
%         \blue{$\Rightarrow \mathit{eval}(P, v_1,\dots,v_n) = \{ \mathit{mapOutput}(\omega, I) | I \in \mathit{AS}(P \cup \mathit{genFacts}(\iota, v_1,\dots,v_n)) \}$}.
%     \end{block}

%     \begin{block}{Language independence}
%         \begin{itemize}
%             \item The specification language is \blue{largely independent} of a concrete OOP language \\
%                 $\Rightarrow$ porting the interpreter library to other OOP languages is easily possible. \\
%                 $\Rightarrow$ the same annotated program can be used with \blue{multiple OOP languages}.
%             \item Currently, we provide a prototypical implementation \blue{py-aspio} for Python.
%         \end{itemize}
%     \end{block}
% \end{frame}

\begin{frame}[allowframebreaks,fragile]{3-Colorability}
    To give an overview over the usage of py-aspio,
    we show an example implementation of the 3-colorability problem:
    \begin{itemize}
        \item Given a graph as input
        \item assign to each node a color (red, green, or blue), such that
        \item any two adjacent nodes have \blue{different} colors.
    \end{itemize}
    \bigskip

    The graph in the Python code is represented by sets of nodes and edges:
    \begin{itemize}
        \item
            Nodes are represented by the class \blue{\lstinline{Node}}, where
            the attribute \blue{\lstinline{label}} is a unique string identifiying the node, and
        \item
            edges are represented by the class \blue{\lstinline{Edge}}, where
            the attributes \blue{\lstinline{first}} and \blue{\lstinline{second}} are the nodes at both ends of the edge.
    \end{itemize}

    \framebreak

    \begin{exampleblock}{coloring.dl (ASP code and input/output specification)}
        \lstinputlisting[basicstyle=\footnotesize]{coloring.dl}
    \end{exampleblock}
    This program can be called with two parameters of types
    \blue{\lstinline{Set<Node>}} and \blue{\lstinline{Set<Edge>}}
    (i.e., any \lstinline{set}-like collection containing instances of the classes \blue{\lstinline{Node}} and \blue{\lstinline{Edge}}, respectively).
    \\
    Its output is of type \blue{\lstinline{Set<ColoredNode>}} (by default, a \lstinline{frozenset} containing instances of \lstinline{ColoredNode}).

    \framebreak

    The Python program first defines the data-holding classes and registers them with py-aspio.
    \begin{exampleblock}{coloring.py (executable Python program)}
        \lstinputlisting[escapechar=`,basicstyle=\footnotesize,language=python]{coloring-part1.py}
    \end{exampleblock}

    \framebreak

    Then the ASP program is loaded and evaluated in different ways.
    \begin{exampleblock}{coloring.py (cont'd)}
        \lstinputlisting[escapechar=`,basicstyle=\footnotesize,language=python]{coloring-part2.py}
    \end{exampleblock}
\end{frame}
%% NOTE: \end{frame} must be at beginning of line if the frame is [fragile]!! (also, nothing else may be on that line)

\begin{frame}{Further examples}
    The 3-Colorability example and others can be found in the ``examples'' directory of the py-aspio repository:
    \url{https://github.com/hexhex/py-aspio/tree/master/examples}
\end{frame}



\section{Input Specification}

\begin{frame}[fragile]{Input Specification}
    Consists of two parts:
    \begin{itemize}
        \item
            The \blue{parameter list} defines what input is expected from the Python program.
        \item
            The \orange{predicate specifications} control how atoms are generated from the input data.
    \end{itemize}

    \bigskip

    In the example from before:
    \begin{lstlisting}[escapechar=`,basicstyle=\footnotesize,autogobble,moredelim={**[is][\color{blue}]{@1}{@}},moredelim={**[is][\color{orange}]{@2}{@}}]
    %!  INPUT (@1Set<Node> nodes, Set<Edge> edges@) {
    %!      @2node(n.label) for n in nodes;@
    %!      @2edge(e.first.label, e.second.label) for e in edges;@
    %!  }
    \end{lstlisting}
\end{frame}

\begin{frame}{Input parameter list}
    A comma-separated list of parameters, each of the form type followed by name, e.g. \blue{\lstinline{int x}} defines a parameter with name \blue{\lstinline{x}} of type \blue{\lstinline{int}}.

    \bigskip

    Available types: \\
    \begin{tabular}{ll}
        \blue{$\mathit{int}$}      & integers \\
        \blue{$\mathit{str}$}      & character strings \\
        \blue{$\mathit{Set} \langle T \rangle$}   & set containing elements of type $T$ \\
        \blue{$\mathit{Sequence} \langle T \rangle$}   & list containing elements of type $T$ \\
        \blue{$\mathit{Dictionary} \langle K, V \rangle$}   & dictionary with keys of type $K$ and values of type $V$ \\
        \blue{$\mathit{Tuple} \langle T_1,\dots,T_n \rangle$}   & $n$-tuple with components of types $T_1,\dots,T_n$ \\
        \blue{$\mathit{MyClass}$}, \dots   & user-defined classes \\
    \end{tabular}

    \bigskip

    In py-aspio, i.e., the Python implementation of the specification language,
    the actual types are not checked at runtime but serve as documentation.
\end{frame}

\begin{frame}[fragile]{Predicate specifications}
    First part defines the form of generated facts.
    Example:
    \begin{lstlisting}[escapechar=`,basicstyle=\footnotesize,moredelim={**[is][\color{blue}]{@1}{@}},moredelim={**[is][\color{orange}]{@2}{@}},moredelim={**[is][\color{darkgreen}]{@3}{@}},moredelim={**[is][\color{red}]{@4}{@}}]
    @1edge@(@2e@@3.first.label@, @2e@@3.second.label@, @2e@@3.someIntArray@@4[3]@)
    \end{lstlisting}
    This will create facts of the form \blue{\lstinline{edge("node1", "node2", 27)}}.

    \medskip
    Available components:
    \begin{itemize}
        \item \blue{ASP predicate}
        \item \orange{Variables}: input parameter or introduced by loop (next slide) % construct %(see next slide)
        \item \darkgreen{Attribute access}
        \item \red{Subscription} (currently only for integer subscripts)
    \end{itemize}

    \medskip
    How are python objects mapped to ASP terms?
    \begin{itemize}
        \item \blue{\lstinline{int}}: Integer constant
        \item \blue{\lstinline{str}}: Quoted string constant (special characters are escaped transparently by py-aspio)
        % \item everything else: Call \blue{str(...)} on the object and then treat it like a string
        \item Everything else is converted to \blue{str} first
    \end{itemize}
\end{frame}

\begin{frame}[fragile]{Iteration}
    Second part of predicate specification allows iteration over \darkgreen{collections}.
    Example:
    \begin{lstlisting}[escapechar=`,mathescape,basicstyle=\footnotesize,moredelim={**[is][\color{blue}]{@1}{@}},moredelim={**[is][\color{orange}]{@2}{@}},moredelim={**[is][\color{darkgreen}]{@3}{@}},moredelim={**[is][\color{red}]{@4}{@}}]
        p(...) for $\orange{x_1}$ in $\darkgreen{y_1}$ ... for $\orange{x_k}$ in $\darkgreen{y_k}$.
    \end{lstlisting}
    % @1edge@(@2e@@3.first.label@, @2e@@3.second.label@, @2e@@3.someIntArray@@4[3]@)

    \orange{Loop variables $x_i$} must be new %(no shadowing allowed)
    and refer to elements
    of \darkgreen{collections $y_i$} \\
    (the $\darkgreen{y_i}$s consist of variables plus attribute access and subscription).

    \medskip
    Evaluated from left to right (same as list comprehensions in Python),
    meaning a variable $\orange{x_i}$ can be used in all $\darkgreen{y_j}$ to the right ($\darkgreen{y_{i+1}},\dots,\darkgreen{y_k}$),
    allowing for nested iteration.

    \medskip
    Value of the \orange{loop variables}:
    \begin{itemize}
        \item If $\darkgreen{y_i}$ is a set, $\orange{x_i}$ will loop over all elements of $\darkgreen{y_i}$.
        \item If $\darkgreen{y_i}$ is a sequence, $\orange{x_i}$ will loop over all $(\mathit{index},\mathit{element})$ pairs of $\darkgreen{y_i}$.
        \item If $\darkgreen{y_i}$ is a dictionary, $\orange{x_i}$ will loop over all $(\mathit{key},\mathit{value})$ pairs of $\darkgreen{y_i}$.
    \end{itemize}

    \medskip
    What do \blue{set}, \blue{sequence}, \blue{dictionary} mean in Python?

    $\rightarrow$ Any classes that implement the abstract base classes \blue{collections.abc.Set}, \blue{collections.abc.Sequence}, \blue{collections.abc.Mapping}, respectively!
\end{frame}

\begin{frame}[fragile]{Shortcuts}
    \begin{itemize}
        \item Iterating over a sequence (e.g., a \blue{list} instance) produces $(\mathit{index},\mathit{element})$ pairs
            $\Rightarrow$ index and element can be accessed with subscripts \blue{\lstinline{x[0]}} and \blue{\lstinline{x[1]}}

        \item Shortcut:
            use list of variables $\orange{x_1,\dots,x_n}$ instead of single iteration variable $\orange{x}$
            $\Rightarrow$ write \blue{\lstinline{for (i,v) in y}} instead of \blue{\lstinline{for x in y}} for a sequence \blue{\lstinline{y}}
            to access components directly

        \item Analogously for dictionaries and $(\mathit{key},\mathit{value})$-pairs.

        \item This shortcut can also be used for \blue{tuples}.
    \end{itemize}
    % Iterating over sequences (e.g., \blue{list} instances) will produce $(\mathit{index},\mathit{element})$-pairs.
    % The index and element can be accessed with subscripts \blue{\lstinline{x[0]}} and \blue{\lstinline{x[1]}}, respectively, where \blue{\lstinline{x}} is the loop variable.

    % Analogously for dictionaries and $(\mathit{key},\mathit{value})$-pairs.

    % To improve readability, it is possible to use a list of variables $\orange{(x_1,\dots,x_n)}$ in place of a single iteration variable $\orange{x}$.
    % For instance, an iteration \blue{\lstinline{for x in y}} over the key-value pairs of the dictionary \blue{\lstinline{y}} may also be written as: \blue{\lstinline{for (k,v) in y}}.

    \begin{example}
        % Consider the following input specification:
        \begin{lstlisting}[basicstyle=\footnotesize,autogobble]
        INPUT (Sequence<Tuple<int, int>> readings) {
            temperature(x[0], x[1][0]) for x in readings;
            humidity(t, hum) for (t,(temp,hum)) in readings;
        }
        \end{lstlisting}
        The specification of \lstinline{temperature} uses subscripts to access the components,
        while the specification of \lstinline{humidity} assigns the values directly to different variables.
    \end{example}

    Additionally, the underscore \blue{\lstinline{_}} serves as \emph{don't care} variable to ignore unneeded values.
\end{frame}



\section{Output Specification}

\begin{frame}[fragile]{Output Specification}
    % The output specification defines how answer sets are mapped back to objects.
    % \medskip
    Enables access to results of the ASP program:

    \begin{itemize}
        \item Each answer set is mapped to a single result object.
        \item The output specification defines one or more attributes for the result object by referring to the answer set.
    \end{itemize}

    \medskip
    Syntactically, it is just a list of assignments of the form ``\blue{name} = \orange{object}''.

    \begin{exampleblock}{Example}
        \begin{lstlisting}[escapechar=`,basicstyle=\footnotesize,autogobble,moredelim={**[is][\color{blue}]{@1}{@}},moredelim={**[is][\color{orange}]{@2}{@}}]
        %!  OUTPUT {
        %!      @1labels@ = @2set { query: label(X); content: X; }@;
        %!  }
        \end{lstlisting}
    \end{exampleblock}
\end{frame}

\begin{frame}{Expressions overview}
    The objects on the right-hand side can be built from the following types of expressions:
    \begin{itemize}
        \item Integer constants, e.g. \blue{0}, \blue{123}, \blue{-5}, \dots
        \item String constants, e.g. \blue{"hello"}, \blue{"one\textbackslash{}ntwo"}, \dots
        \item Tuples: \blue{$(e_1,e_2,\dots,e_n)$} with subexpressions $e_1,\dots,e_n$
        \item Instantiating user-defined classes: \blue{$\textit{MyClass}\,(e_1,e_2,\dots,e_n)$}
        \item Sets: \blue{$\text{set}~\{~\text{query:}~q;~\text{content:}~e;\}$}
        \item Shortcuts for sets: \blue{$\text{set}~\{~p/n~\}$} and \blue{$\text{set}~\{~p/n~\lstinline{->}~\textit{MyClass}~\}$}
        \item Sequences/lists: \blue{$\text{sequence}~\{~\text{query:}~q;~\text{index:}~i;~\text{content:}~e;\}$}
        \item Dictionaries: \blue{$\text{dictionary}~\{~\text{query:}~q;~\text{key:}~k;\text{content:}~e;\}$}
        \item Variables (after being introduced by a \blue{query}): \blue{X}, \blue{Y}, \blue{Color}, \dots
        \item References: \blue{\&\textit{name}}
    \end{itemize}
    % \bigskip
    % More details on the following slides.
\end{frame}

\begin{frame}[fragile,allowframebreaks]{Collection Expressions}
    Collections expressions use a query $q$ to access the answer set. \\
    There are three types:
    \begin{itemize}
        \item \blue{$\text{set}~\{~\text{query:}~q;~\text{content:}~e;\}$}
        \item \blue{$\text{sequence}~\{~\text{query:}~q;~\text{index:}~i;~\text{content:}~e;\}$}
        \item \blue{$\text{dictionary}~\{~\text{query:}~q;~\text{key:}~k;\text{content:}~e;\}$}
    \end{itemize}

    \medskip
    The query $q$ is a list of (possibly non-ground) ASP literals that are to be matched against the answer set.
    Any variables introduced by (non-ground literals in) the query $q$ can be used in the expressions $e$, $i$, $k$.

    Every match of the query yields an element of the collection, with any ASP variables introduced in $q$ replaced by their matched values.
    \begin{itemize}
        \item The \blue{content} $e$ is an expression that defines the elements of the collection.
        \item In case of a sequence, the \blue{index} $i$ (must be a variable) determines the position of the element.
            The values of $i$ over all matches must form a consecutive range of integers without gaps or duplicates.
        \item In case of a dictionary, the \blue{key} $k$ determines the key of the element.
            As $e$, the key can be any expression.
    \end{itemize}

    Note that in Python, set elements and dictionary keys must be \href{https://docs.python.org/3/reference/datamodel.html#object.__hash__}{hashable}.

    \framebreak

    \begin{example}
        Consider the following output specification:
        \begin{lstlisting}[basicstyle=\footnotesize,autogobble]
        OUTPUT {
            labels = set { query: vertex(X); content: X; };
            red_nodes = set { query: color(X, red); content: X; };
        }
        \end{lstlisting}
        It defines two attributes, \blue{\lstinline{labels}} and \blue{\lstinline{red_nodes}},
        whose concrete values depend on the current answer set.
        Now consider the answer set
        \begin{align*}
            I = \{ & \mathit{vertex}(a), \mathit{vertex}(b), \mathit{vertex}(c), \mathit{edge}(a,b), \mathit{edge}(a,c), \\
                    & \mathit{color}(a, \mathit{blue}), \mathit{color}(b, \mathit{red}), \mathit{color}(c, \mathit{red}) \}.
        \end{align*}
        % For this answer set:
        % \begin{itemize}
        %     \item
                The query \blue{\lstinline{vertex(X)}} matches all atoms of the predicate $\mathit{vertex}$,
                thus \blue{\lstinline{labels}} will have the value \lstinline[language=python]|{"a", "b", "c"}|.
            % \item

                The query \blue{\lstinline{color(X, red)}} has a constant as second argument,
                which means only atoms with this value are matched, here: $\mathit{color}(b, \mathit{red})$ and $\mathit{color}(c, \mathit{red})$.
                The value of \blue{\lstinline{red_nodes}} will be \lstinline[language=python]|{"b", "c"}|.
        % \end{itemize}
    \end{example}

    \framebreak
    \begin{itemize}
        \item Since the query refers to elements of the ASP code, the same naming conventions apply:
            words starting with an upper-case letter are variables,
            and words starting with a lower-case letter are constant symbols.
            % variables start with an upper-case letter, and constants 
        \item Note that for consistency, all variable values are strings.
            If an integer value is required, use \blue{\lstinline{int(X)}} instead of \blue{\lstinline{X}}.
            The only exception is the \blue{index} clause in sequence expressions, which is always interpreted as an integer.
        \item
            As in ordinary ASP, anonymous variables \blue{\lstinline{_}} can be used in the query.
    \end{itemize}

    \framebreak

    \begin{exampleblock}{Nested collections}
        Collection expressions can be nested.
        The following extracts a dictionary that maps colors to the set of nodes with that color:
        \begin{lstlisting}[basicstyle=\footnotesize,autogobble]
        %!  OUTPUT {
        %!      labels_by_color = dictionary {
        %!          query: color(_, C);
        %!          key: C;
        %!          content: set { query: color(L, C); content: L; };
        %!      };
        %!  }
        \end{lstlisting}
        If this expression is evaluated under the answer set
        \[ \{ \mathit{color}(a, \mathit{blue}), \mathit{color}(b, \mathit{red}), \mathit{color}(c, \mathit{red}) \}, \]
        the dictionary \blue{\lstinline{labels_by_color}} yields the mappings $\blue{\mathit{blue} \mapsto \{a\}}$ and $\blue{\mathit{red} \mapsto \{b,c\}}$.
        Note that the variable~\blue{C} is introduced in the dictionary expression.
        For every match of its query \blue{\lstinline{color(_, C)}},
        the nested set expression is evaluated with \blue{C} fixed to the matched value,
        thus generating a set of labels colored by the color~\blue{C}.
    \end{exampleblock}
\end{frame}

\begin{frame}[fragile]{Shortcuts for Set Expressions}
    % \begin{lstlisting}[escapechar=`,basicstyle=\footnotesize,autogobble,moredelim={**[is][\color{blue}]{@1}{@}},moredelim={**[is][\color{orange}]{@2}{@}}]
    % OUTPUT {
    %     vertices = set { query: vertex(X); content: X; };
    %     edges = set { query: edge(X, Y); content: (X, Y); };
    %     graph = Graph(&vertices, &edges);
    % }
    % \end{lstlisting}

    \begin{itemize}
        \item
            \blue{$\text{set}~\{~p/n~\}$}
            stands for
            \blue{$\text{set}~\{~\text{query:}~p(X_1,\dots,X_n);~\text{content:}~(X_1,\dots,X_n);~\}$}
        \item
            \blue{$\text{set}~\{~p/n~\lstinline{->}~\textit{MyClass}~\}$}
            stands for \\
            \blue{$\text{set}~\{~\text{query:}~p(X_1,\dots,X_n);~\text{content:}~\textit{MyClass}(X_1,\dots,X_n);~\}$}
    \end{itemize}

    \begin{example}
        \begin{lstlisting}[basicstyle=\footnotesize,autogobble]
        OUTPUT {
          labels = set { vertex/1 };
          colored_nodes = set { color/2 -> ColoredNode };
        }
        \end{lstlisting}
        is equivalent to
        \begin{lstlisting}[basicstyle=\footnotesize,autogobble]
        OUTPUT {
          labels = set { query: vertex(X); content: X; };
          colored_nodes = set { query: color(X, Y); content: ColoredNode(X, Y); };
        }
        \end{lstlisting}
    \end{example}
\end{frame}



\section{Python Interface}

\begin{frame}[fragile]{Loading the ASP Program}
    ASP programs are represented by the class \blue{Program}.

    % Load programs from files or strings:
    Programs can be loaded from files or strings:
    \begin{lstlisting}[basicstyle=\footnotesize,language=python]
    import aspio
    prog1 = aspio.Program(filename='aspfile.dl')
    prog2 = aspio.Program(code=r'''
        a :- not b.
        b :- not a.
    ''')
    prog3 = aspio.Program()  # empty program
    \end{lstlisting}

    To merge ASP code from multiple sources:
    \begin{lstlisting}[basicstyle=\footnotesize,language=python]
    prog.append_file('anotherfile.dl')
    prog.append_code('c :- b.')
    \end{lstlisting}

    The input and output specifications are extracted automatically.
    However, note that one program may contain \blue{at most one input specification} and \blue{at most one output specification}.
\end{frame}

\begin{frame}{Evaluating the ASP Program}
    Two methods for evaluating a program $p$:
    \begin{itemize}
        \item \blue{prog.solve(\dots)}:
            Returns an iterable allowing to iterate over \blue{all answer sets} of the program.
            Note that due to the semantics of ASP, the concrete order of the returned objects has no meaning!
        \item \blue{prog.solve\textunderscore{}one(\dots)}:
            Returns the \blue{first answer set} found by the solver, or \blue{\lstinline{None}} if the program does not have any answer sets.
    \end{itemize}

    The positional arguments of these methods are the parameters defined by the input specification.

    The answer sets are represented by instances of the class \blue{Result}.
    These objects have attributes as defined by the output specification (see next slide).

    \bigskip
    Additional keyword arguments:
    \begin{itemize}
        \item \blue{options}: specify additional options to pass to the solver (more details later).
        \item \blue{solver}: specify the ASP solver to use. Currently, only \href{http://www.kr.tuwien.ac.at/research/systems/dlvhex/}{dlvhex} is supported.
    \end{itemize}
\end{frame}

\begin{frame}[fragile]{Working with the Output of an ASP Program}
    % Consider the input/output specifications of the 3-Colorability example from the introduction (with parts omitted):
    Consider the annotations of the 3-Colorability example (with parts omitted):
    \begin{lstlisting}[basicstyle=\footnotesize,autogobble]
    %!  INPUT (Set<Node> nodes, Set<Edge> edges) { ... }
    %!  OUTPUT { colored_nodes = set { ... }; }
    \end{lstlisting}
    % Consider the output specification of the 3-Colorability example (part omitted):
    % \begin{lstlisting}[basicstyle=\footnotesize,autogobble]
    % %! OUTPUT { colored_nodes = set { ... }; }
    % \end{lstlisting}

    According to the input specification, the evaluation requires two arguments.

    \begin{itemize}
        \item
            Use the method \blue{\lstinline{solve}} to iterate over all answer sets.
            The collection returned by \blue{solve} contains one object per answer set,
            with attributes as defined in the output specification:
            \begin{lstlisting}[basicstyle=\footnotesize,autogobble,language=python]
            for result in prog.solve(nodes, edges):
                print(result.colored_nodes)
            \end{lstlisting}
            A shortcut if only one attribute is needed (note the required prefix \blue{\lstinline{each_}}):
            \begin{lstlisting}[basicstyle=\footnotesize,autogobble,language=python]
            for cns in prog.solve(nodes, edges).each_colored_nodes:
                print(cns)
            \end{lstlisting}
        \item
            Use the method \blue{\lstinline{solve_one}} if only a single arbitrary answer set is needed,
            or if only the consistency of the program is of interest:
            \begin{lstlisting}[basicstyle=\footnotesize,autogobble,language=python]
            result = prog.solve_one(nodes, edges)
            if result is not None: print(result.colored_nodes)
            else: print("no answer set exists")
            \end{lstlisting}
    \end{itemize}
\end{frame}

\begin{frame}[fragile]{Solver Options}
    Instance of class \blue{SolverOptions} with the following attributes:
    \begin{itemize}
        \item \blue{\lstinline{max_answer_sets}}:
            Set to an integer $n$ to compute at most $n$ answer sets,
            or to \blue{None} to compute all (default: None).
        \item \blue{\lstinline{max_int}}:
            Set maximum integer value (may be necessary for arithmetic).
        \item \blue{\lstinline{capture}}:
            A list of strings; the names of predicates that should be captured
            (in addition to the output specification).
            The captured data is available through the attribute \blue{\lstinline{answer_set}} of the class \blue{Result} (cf.~below example).
        \item \blue{\lstinline{custom}}:
            A list of strings; any custom options that are to be passed to the solver subprocess.
    \end{itemize}
    \begin{example}%{Example}
        The following snippet show how to evaluate an ASP program with the \lstinline{max_int} option set to 25,
        and accessing the raw data of predicate p:
        \begin{lstlisting}[basicstyle=\footnotesize,language=python,autogobble]
            prog = aspio.Program(...)
            opts = aspio.SolverOptions(max_int=25, capture=['p'])
            for result in prog.solve(input1, input2, options=opts):
                ps = result.answer_set['p']  # type: Iterable[Tuple[str, ...]]
        \end{lstlisting}
    \end{example}
\end{frame}

\begin{frame}[fragile]{User-defined Classes in Output}
    User-defined classes that are to be used in an output specification, either
    \begin{itemize}
        \item must be fully qualified (e.g., \blue{package.module.MyClass}), or
        \item have to be \blue{registered with py-aspio} as described below.
    \end{itemize}

    \medskip
    There are three methods to register classes for a program \lstinline{prog}:
    \begin{itemize}
        \item
            Pass the constructor and its name to the method \blue{register}:
            \begin{lstlisting}[basicstyle=\footnotesize,language=python,autogobble]
            prog.register(MyClass, 'MyClass')
            \end{lstlisting}
        \item
            If the name is missing, the constructor's attribute \blue{\lstinline{__name__}} is used:
            \begin{lstlisting}[basicstyle=\footnotesize,language=python,autogobble]
            prog.register(MyClass)
            \end{lstlisting}
        \item
            Or simply register all callable objects in the current global scope:
            \begin{lstlisting}[basicstyle=\footnotesize,language=python,autogobble]
            prog.register_dict(globals())
            \end{lstlisting}
    \end{itemize}

    \medskip
    Alternatively, these methods may be called on the \blue{aspio} module
    in order to register classes globally (i.e., for all ASP programs in the current application):
    \begin{lstlisting}[basicstyle=\footnotesize,language=python,autogobble]
    import aspio
    aspio.register(MyClass)
    \end{lstlisting}
\end{frame}

\begin{frame}[allowframebreaks]
    \frametitle{References}

    \footnotesize
    \bibliographystyle{apalike}
    \bibliography{using_py-aspio}
\end{frame}

\end{document}
